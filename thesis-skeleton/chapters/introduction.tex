\chapter{Introduction}
\label{chap:intro}

\begin{itemize}
	\item Domain: Pull request mechanism as central element to contribute to a repository \autocite{DBLP:conf/icse/TsayDH14}, signifying a request to to merge proposed changes with the existing codebase
	\item Problem: Lack of transparency and enforcement regarding contribution guidelines resulting in complex interactions between developers, maintainers and new users i.e., following repository guidelines requires more effort than following own approach e.g., finding a suitable reviewer, which tasks are to be completed, what is the time frame for one pull request? \ac{cicd} pipelines help to automate some aspects but introduce further complexity.
	\item Method: Extract source code and pull request data from GitHub to form an \ac{ocel} for analysis with regards to contribution process and code quality
	\item Result: Workflow diagram focusing on pull request process and evolution of code quality makes process clear and easy to understand, helping maintainers to optimize workflows and new developers adapt the desired process. 
	\item Evaluation: Case study of using \ac{ocel} as single source of truth through analysis of contribution processes and code quality in three large Python repositories to visualize code quality from different viewpoints and detect guideline changes as well as comparing their effectiveness.
\end{itemize}

\section{Motivation}
\label{sec:intro_ssec:motiv}
\begin{itemize}
	\item It is essential to follow guidelines (naming conventions, process of finding reviewers \& expected time until merge), in order to contribute code to a repository effectively
	\item Documentation how contributions are handled can hardly capture all scenarios
	\item Specification of scenario, where developer finds a bug during use of a program and wants to contribute own fix to repository without prior open-source contribution experience (i.e., a new developer)
\end{itemize}

\paragraph{Problem Relevance.} 
\begin{itemize}
	\item Increasing popularity of open-source projects in contrast to lack of new developers to maintain repositories \autocite{DBLP:journals/corr/abs-2208-04895}\autocite{DBLP:journals/ese/RehmanWKIM22}
	\item No known implementation of data extraction of pull request information from Git repositories to form \ac{ocel}
	\item Git commit and pull request data is hard to overview making visualizations valuable, especially for new developers
	\item Contribution guidelines as initial point of contact \autocite{DBLP:conf/icsm/ElazharySEZ19} lack step-by-step walk-through of contribution process
	\item Different developers pursuing their own goals can cause worsening of overall code quality, especially with new developers
\end{itemize}

\paragraph{Existing Solutions.}
\begin{itemize}
	\item For developers, both new and experienced, understanding the employed contribution process is essential, therefore tools helping developers and maintainers to make better decisions. Tools excelling at measuring code quality, analyzing the contribution process or helping reviewers decide whether to approve a pull request exist separately, the aim of this thesis is to combine aspects about contribution process visualization and code quality measurement to unveil their relationship.
	\item Data extraction with data only from git (local repository) and \ac{restapi} (remote repository) exist separately, but no combined approach could be found at the point of writing.
	\item Code quality metrics have existed for a long time, they have been refined and new metrics have been proposed \autocite{DBLP:journals/smr/CodabuxSC24}, but there has been little focus on code quality evolution $\Rightarrow$ Use \ac{ocel} to visualize the change of code quality. Code review, an important aspect of contribution processes, and its impact on code quality has been studied, but again the wholistic view of the contribution process was not analyzed \autocite{DBLP:journals/ese/McIntoshKAH16}
	% CI/CD aspect could also be mentioned here
	\item Work by \autocite{DBLP:conf/icsm/ElazharySEZ19} analyzes adherence to contribution guidelines and detects anomalies in the contribution process, but the impact of contribution processes on code quality is not discussed.
\end{itemize}


\section{Problem Statement}
\label{sec:intro_ssec:probs}
\begin{itemize}
	\item Datasets for \ac{msr} are specific to one viewpoint, lacking the analysis of interdependence between different aspects of contributing and often are specific to an older version of process mining format like the original \ac{ocel} specification
	\item Existing code quality tools focus too little on the changes of code quality during the history of a repository and often just analyze one perspective
	% Rewrite to focus on specific tools to compare to like CodeScene
	\item The effect of adhering to contribution guidelines is largely unclear
\end{itemize}


\section{Research Questions}
\label{sec:intro_ssec:rqs}
\begin{itemize}
	\item How can local and remote repository data be standardized and combined into an \ac{ocel}?
	\begin{itemize}
		\item How is Git data extracted from open source software repositories?
		\item Which additional benefits does an \ac{ocel} offer compared to traditional event logs in this domain?
		\item Why is this definition and format of \ac{ocel} chosen?
	\end{itemize}
	\item How does code quality evolve, when inspected from different viewpoints?
	\begin{itemize}
		\item Does code quality generally increase within a repository?
		\item Do maintainability index and Pylint score follow the same trends?
		\item How does code quality evolve from a file, author, pull request or repository viewpoint?
	\end{itemize}
	\item Does adhering to contribution guidelines result in the desired code quality?
	\begin{itemize}
		\item Are changes in contribution guidelines noticeable in the evolution of code quality?
		\item Do deviations from guidelines affect code quality?
		\item Which deviations from guidelines have the biggest impact on code quality?
	\end{itemize}
\end{itemize}

\section{Research Goals}
\label{sec:intro_ssec:rgs}
\begin{itemize}
	\item Create an approach to build datasets of contribution processes in GitHub repositories incorporating multiple viewpoints on code quality
	\item Visualize evolution of code quality from different (repository, pull request, author and file) viewpoints solely based on \ac{ocel}
	\item Pinpoint most common deviations from guidelines through conformance checking 
	\item Compare discovered pull request process models and their impact on code quality
\end{itemize}


\section{Contributions}
\label{sec:intro_ssec:c}
\begin{itemize}
	\item General approach to build datasets on GitHub contribution data in an \ac{ocel} format
	\item Visualization tool to build code quality graphs based on built \acp{ocel}
	\item Case study of mining three large Python repositories and their contribution guidelines
\end{itemize}


\section{Thesis Structure}
\label{sec:intro_ssec:ts}
\begin{enumerate}
	\item First discuss fundamental principles about code quality and \ac{ocpm} 
	\item Related work on code quality metrics and contribution process mining
	\item Design decisions in general approach from extraction to analysis
	\item Evaluating the proposed approach through a case study
	\item Concluding this thesis with an outlook for future work
\end{enumerate}
