\chapter{Preliminaries}
\label{chap:prelim}
\begin{itemize}
	\item The following sections discuss fundamentals about code quality metrics, what contributions are in the context of open-source development and how \acp{ocel} are structured.
\end{itemize}

\section{Introduction of Running Example}
\begin{itemize}
	\item Toy repository with contribution guidelines, pull request template, protected main branch, 3 contributors, new user, two pull requests, multiple reviews per pull request, based on scenario mentioned in \autoref{sec:intro_ssec:motiv}
\end{itemize}

\begin{tabular}{|l|l|l|l|l|l|l|}
\hline
$e_i$ & activity & time & pull\_request & commit & user & file \\
\hline
$c\_01$ & commit & 25-07-05 21:41:34 & $\{1,3\}$ & $\{1234567\}$ & $\{\text{alex}\}$ & $\{\text{contrib.md, index.py}\}$ \\ 
\hline
\end{tabular}



\section{Definition of Code Quality Metrics}
\begin{itemize}
	\item Definition of chosen maintainability index variant based on same file as above in running example
	\begin{itemize}
		\item Definition of cyclomatic complexity: How many conditional branches does method, file or class have, by using changed file of running example
		\item Definition of Halstead's complexity measures using same file in running example
	\end{itemize}
	\item Explanation of \emph{Pylint} score: Linting and variable names using same file in running example
	\item Definition of how file code quality data is aggregated for author, pull request and repository viewpoints
\end{itemize}

\section{Definition of Contribution}
\begin{itemize}
	\item Which actions are considered as contribution on GitHub and which aspects of contribution are considered in this thesis i.e., actions affecting code quality or contribution process with respect to committing
	\item Differentiation between source code and documentation pull requests for later analysis of their impact on code quality and contribution guidelines respectively
\end{itemize}

\section{Definitions of Object-Centric Process Mining Related Terms}
\begin{itemize}
	\item Definition of term viewpoint using above example
	\item Explain, what an event vs.\ activity vs.\ event type is by using data from above example \ac{ocel} % Check with papaer from Wil (reference) if definitions match
 	\item Definition of variant of \ac{ocel} % Check Haukes choice of definition
\end{itemize}