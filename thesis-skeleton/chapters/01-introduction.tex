\chapter{Introduction}
\label{chap:intro}

\begin{itemize}
	\item Domain: Pull request mechanism as central element to contribute to a repository \autocite{DBLP:conf/icse/TsayDH14}
	\item Problem: Lack of transparency and enforcement regarding contribution guidelines resulting in complex interactions between developers, maintainers and new users i.e., following repository guidelines requires more effort than following own approach
	\item Solution: By extracting source code and pull request data from GitHub to form an \ac{ocel} for analysis the pull request process and evolution of code quality are evaluated, clarifying the relationship between the contribution process and code quality, which is demonstrated in a case study of three large Python repositories

\end{itemize}

\section{Motivation}
\label{sec:intro_ssec:motiv}
\begin{itemize}
	\item It is essential to follow guidelines (naming conventions, process of finding reviewers \& expected time until merge), in order to contribute code to a repository effectively
	\item Documentation how contributions are handled can hardly capture all scenarios
	\item Specification of scenario, where developer finds a bug during use of a program and wants to contribute own fix to repository for illustrating status-quo
\end{itemize}
\paragraph{Problem Relevance.} 
\begin{itemize}
	\item Increasing popularity of open-source projects in contrast to lack of new developers to maintain repositories \autocite{DBLP:journals/corr/abs-2208-04895}\autocite{DBLP:journals/ese/RehmanWKIM22}
	\item No known implementation of data extraction of pull request information from Git repositories to form \ac{ocel}
	\item Git commit and pull request data is hard to overview making visualizations valuable
	\item Contribution guidelines as initial point of contact \autocite{DBLP:conf/icsm/ElazharySEZ19} lack step-by-step walk-through of contribution process
	\item Increased code quality motivates the adherence to contribution guidelines
\end{itemize}

\paragraph{Existing Solutions.}
\begin{itemize}
	\item Data extraction with data only from git (local repository) and \ac{restapi} (remote repository) exist separately, but no combined approach could be found at the point of writing.
	\item Code quality metrics have existed for a long time, they have been refined, but there has been little focus on code quality evolution $\Rightarrow$ Use \ac{ocel} to visualize the change of code quality
	\item Work by \autocite{DBLP:conf/icsm/ElazharySEZ19} analyzes adherence to contribution guidelines, but excludes impact of process mining to improve contribution guidelines and code quality
\end{itemize}


\section{Problem Statement}
\label{sec:intro_ssec:probs}
\begin{itemize}
	\item Datasets for \ac{msr} are outdated and specific to one viewpoint, lacking the interdependence of different aspects of contributing
	\item Existing tools focus too little on the changes of code quality during the history of a repository
	\item The effect of adhering to contribution guidelines is largely unclear
\end{itemize}


\section{Research Questions}
\label{sec:intro_ssec:rqs}
\begin{itemize}
	\item How can local and remote repository data be standardized and combined into an \ac{ocel}?
	\begin{itemize}
		\item How is Git data extracted from open source software repositories?
		\item Which additional benefits does an \ac{ocel} offer compared to traditional event logs in this domain?
		\item Why is this definition and format of \ac{ocel} chosen?
	\end{itemize}
	\item How does code quality evolve, when inspected from different viewpoints?
	\begin{itemize}
		\item Does code quality generally increase within a repository?
		\item Do maintainability index and Pylint score follow the same trends?
		\item How does code quality evolve from a file, author or pull request viewpoint?
	\end{itemize}
	\item Does adhering to contribution guidelines result in the desired code quality?
	\begin{itemize}
		\item Are changes in contribution guidelines noticeable in the evolution of code quality?
		\item Do deviations from guidelines affect code quality?
		\item Which deviations from guidelines have the biggest impact on code quality?
	\end{itemize}
\end{itemize}

\section{Research Goals}
\label{sec:intro_ssec:rgs}
\begin{itemize}
	\item Combine data of local and remote repository data from GitHub to form an \ac{ocel} incorporating code quality
	\item Visualize evolution of code quality from different (repository, pull request, author and file) viewpoints solely based on \ac{ocel}
	\item Pinpoint most common deviations from guidelines through conformance checking 
	\item Compare discovered pull request process models and their impact on code quality
\end{itemize}


\section{Contributions}
\label{sec:intro_ssec:c}
\begin{itemize}
	\item Extraction pipeline to combine local and remote data from GitHub in one \ac{ocel} for process mining
	\item Visualization tool to build code quality graphs based on \acp{ocel}
	\item Case study of mining three large Python repositories and their contribution guidelines
\end{itemize}


\section{Thesis Structure}
\label{sec:intro_ssec:ts}