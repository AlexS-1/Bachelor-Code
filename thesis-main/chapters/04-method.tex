\chapter{Method}
\label{chap:method}
Overview of approach
\begin{itemize}
	\item Source code of commits is analyzed for maintainability index and Pylint score
	\item Data extracted from local git repository is related to objects from remote repository before being unified as \ac{ocel}
	\item \ac{ocel} is analyzed for contribution guidelines changes and split into separate logs for respective process discovery and conformance checking to measure impact of guidelines on code quality
\end{itemize}

\section{Extraction of Code Quality Metrics and Contribution Workflows}
To answer RQ1:
\begin{itemize}
% Summarize key aspects of workflow in graphic similar to Pull Request Decisions Explained

	\item \emph{PyDriller} is suitable to extract source code evolution by commits
	\item Source code is analyzed with Pylint\footnote{\url{https://pylint.pycqa.org/} (accessed 2025-05-29)} and Radon\footnote{\url{https://radon.readthedocs.io/} (accessed 2025-05-29)}
\end{itemize}

%TODO Integrate in storyliene

\textbf{Definition of Contribution Activities}
\begin{itemize}
	\item Which actions are considered as contribution on GitHub and which aspects of contribution are considered in this thesis i.e., actions affecting code quality or contribution process with respect to committing
	\item Differentiation between source code and documentation pull requests for later analysis of their impact on code quality and contribution guidelines respectively
\end{itemize}

In the context of this thesis, contributions refer to user actions that interact with a repository to influence its content, structure, or discussion. Following GitHub's common terminology, these include creating repositories, forking, committing code, opening issues or discussions, and submitting or reviewing pull requests.

We primarily focus on contributions that are logged and measurable within the GitHub event and commit data, particularly pull request interactions, commits, reviews, and merges. These events are captured in the object-centric event log and discussed in detail later.

\begin{itemize}
	\item \emph{GitHub \ac{restapi}} is used for filling \ac{ocel} with events via issue timeline \ac{api} \footnote{\url{https://docs.github.com/en/rest/issues/timeline?apiVersion=2022-11-28} (accessed 2025-05-29)}
	\item The pull request object is used as central part for connecting code quality with workflow analysis through process mining
	\item Matching local user's names with remote usernames and renaming files are two challenges when relating objects with events
\end{itemize}

\section{Dissecting Data into Blocks for Analysis}
To answer RQ2:
\begin{itemize}
	\item Document based database is ideal for storing \ac{api} responses in JSON format
	\item Data from \emph{PyDriller} and remote repository are stored in  collections in MongoDB\footnote{\url{https://www.mongodb.com/docs/} (accessed 2025-05-29)}
	\item Code quality is calculated for files and commits to calculate other viewpoints later
	\item Changes in contribution guidelines are detected through code changes on non-source-code files especially on contribution.md and pull requests with respective tags
	\item Event logs are flattened for e.g., pull request as case notion with implementation of flattening algorithm for visualizing process workflows from different viewpoints
\end{itemize}

\section{Visualization of Process Mining Results Suitable for Developers}
To answer RQ3:
\begin{itemize}
	\item Process models are discovered for time frames where contribution guidelines do not change
	\item Process models are evaluated based on effect on code quality in the repository
	\item Code quality can be visualized from file, user, pull request and repository viewpoint through object-centric nature of \ac{ocel}
\end{itemize}