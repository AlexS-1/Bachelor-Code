\chapter{Method}
\label{chap:method}
Overview of approach
\begin{itemize}
	\item Source code of commits is analyzed for maintainability index and Pylint score
	\item Data extracted from local git repository is related to objects from remote repository before being unified as \ac{ocel}
	\item \ac{ocel} is analyzed for contribution guidelines changes and split into separate logs for respective process discovery and conformance checking to measure impact of guidelines on code quality
\end{itemize}

\section{Extraction of Data For Code Quality Analysis and Contribution Data}
\begin{itemize}
	\item \emph{PyDriller} is suitable to extract source code evolution by commits
	\item Source code is analyzed with Pylint\footnote{\url{https://pylint.pycqa.org/} (accessed 2025-05-29)} and Radon\footnote{\url{https://radon.readthedocs.io/} (accessed 2025-05-29)}
	\item \emph{GitHub \ac{restapi}} is used for filling \ac{ocel} with events via issue timeline \ac{api} \footnote{\url{https://docs.github.com/en/rest/issues/timeline?apiVersion=2022-11-28} (accessed 2025-05-29)}
	\item The pull request object is used as central part for connecting code quality with workflow analysis through process mining
	\item Matching local user's names with remote usernames and renaming files are two challenges when relating objects with events
\end{itemize}

\section{Dissecting Data into Blocks for Analysis}
\begin{itemize}
	\item Document based database is ideal for storing \ac{api} responses in JSON format
	\item Data from \emph{PyDriller} is stored in objects collection in MongoDB\footnote{\url{https://www.mongodb.com/docs/} (accessed 2025-05-29)}
	\item Remote data from GitHub is stored in objects and events collections in MongoDB
\end{itemize}

\section{Visualization of Process Mining Results Suitable for Developers}
\begin{itemize}
	\item Changes in contribution guidelines are detected through code changes on non-source-code files especially on contribution.md and pull requests with respective tags
	\item Process models are discovered for time frames where contribution guidelines do not change
	\item Process models are evaluated based on effect on code quality in the repository
	\item Code quality can be visualized from file, commit, pull request and repository viewpoint through object-centric nature of \ac{ocel}
\end{itemize}