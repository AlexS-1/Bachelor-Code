\chapter{Discussion}
\label{chap:discussion}

\section{Data Model of Objects and Events}
\begin{itemize}
	\item Focus on four viewpoints: Repository, pull request, user and file viewpoint
	\item Necessity of the commit object without own viewpoint
	\item Naming conventions used for object-to-object and object-to-event relationships
	\item Important attributes of objects and events including initialization timestamp (1970-01-01 vs.\ actual date of creation) and considerations for merging data from two sources
	\item Object attributes modeled as relationship to other objects
	\item Why were maintainability index and Pylint chosen for code quality analysis
	\item Distinguishing between events and objects: Example of the commit object and event and their differences
\end{itemize}

\section{Considerations for Comparing Code Quality Between Repositories}
\begin{itemize}
\item Settings used for Pylint and their influence on the resulting score
\item Effect of Halstead metrics and cyclomatic complexity on the maintainability index
\item Other code quality aspects not directly accounted for in chosen metrics e.g., code churn metric captured by file viewpoint on code quality
\end{itemize}