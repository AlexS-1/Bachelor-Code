\chapter{Preliminaries}\label{chap:prelim}
The following sections discuss fundamentals about code quality metrics and how \acp{ocel} are structured and used.

\section{Definition of Code Quality Metrics}\label{sec:code-quality-metrics}
To evaluate the impact of contribution processes on code quality, we introduce the following code quality metrics, especially for files modified during commits. These metrics are projected onto event logs via attributes extracted from file-level static analysis tools (e.g., \texttt{pylint}, \texttt{radon}).

\begin{definition}[Cyclomatic Complexity]\label{def:cyclomatic-complexity}
Let the cyclomatic complexity of a program be the number of the linearly independent paths through said program and let the cyclomatic complexity of a section of source code with multiple program entry points i.e., a source code file with multiple method definitions be defined as the sum of the cyclomatic complexities for each program entry point
%TODO Check if 1 + sum(cc_per-method - 1) or sum(cc_per-method) in radon
\end{definition}

\begin{definition}[Halstead Complexity Metrics]\label{def:halstead}
For a section of source code let
	\begin{itemize}
		\item $\theta_1$ := the number of distinct operators
		\item $\theta_2$ := the number of distinct operands
		\item $N_1$ := the total number of operators
		\item $N_2$ := the total number of operands
	\end{itemize}
Thus the following metrics can be calculated
	\begin{itemize}
		\item $\theta = \theta_1 + \theta_2$ = program vocabulary
		\item $\N = N_1 + N_2$ = program length
		\item $V = N \cdot log_2(\theta)$ = Volume 
	\end{itemize}
\end{definition}

\begin{definition}[Maintainability Index]\label{def:maintainability-index}
Using the definitions of \ac{cc} in Definition \autoref{def:cyclomatic-complexity} and V in Definition \autoref{def:halstead}, let the maintainability index $\mathit{MI}_{raw} \in \mathbb{R}$ be defined as:

\[
\mathit{MI}_{raw} = 171 - 5.2 \cdot \ln(\mathit{V}) - 0.23 \cdot \mathit{cc} - 16.2 \cdot \ln(\mathit{loc})
\]

The final Maintainability Index $mathit{MI} \in [0, 100]$ is then scaled to the interval $[0, 100]$ as follows:

\[
\mathit{MI} = \min\left(100, \max\left(0, \frac{100 \cdot \mathit{MI}_{raw}}{171}\right)\right)
\]	
\end{definition}

\section{Definitions of Object-Centric Process Mining Related Terms}\label{sec:ocpm-related-terms}
\subsection{Basics}\label{ssec:basics}
We use the following general notation to express operations on sets.

\begin{definition}[Power Set]\label{def:power-set}
Let $A$ be a set. Then we write $\mathcal{P}(A) := \{ B \mid B \subseteq A \}$ for the power set of $A$.
\end{definition}

\begin{definition}[Union Over an Index Set]\label{def:indexed-union}
Let $I$ be an index set and let $\{ A_i \}_{i \in I}$ be a family of sets indexed by $I$. Then we define the union over the index set as:
\[
\bigcup_{i \in I} A_i := \{ x \mid \exists i \in I: x \in A_i \}.
\]
\end{definition}

\subsection{Universes}\label{ssec:universes}

We recall the described universes in the \ac{ocel} 2.0 specification \autocite{DBLP:journals/corr/abs-2403-01975}:

\begin{definition}[\ac{ocel} Universes]\label{def:ocel-universes}
Let $\mathbb{U}_\Sigma$ be the universe of strings. We define the following pairwise disjoint universes:
\begin{itemize}
  \item $\mathbb{U}_{ev} \subseteq \mathbb{U}_\Sigma$ is the universe of events.
  \item $\mathbb{U}_{etype} \subseteq \mathbb{U}_\Sigma$ is the universe of event types (i.e., activities).
  \item $\mathbb{U}_{obj} \subseteq \mathbb{U}_\Sigma$ is the universe of objects.
  \item $\mathbb{U}_{otype} \subseteq \mathbb{U}_\Sigma$ is the universe of object types.
  \item $\mathbb{U}_{attr} \subseteq \mathbb{U}_\Sigma$ is the universe of attribute names.
  \item $\mathbb{U}_{val}$ is the universe of attribute values.
  \item $\mathbb{U}_{time}$ is the universe of timestamps (with $0 \in \mathbb{U}_{time}$ as the smallest and $\infty \in \mathbb{U}_{time}$ as the largest element).
  \item $\mathbb{U}_{qual} \subseteq \mathbb{U}_\Sigma$ is the universe of qualifiers.
\end{itemize}

Timestamps are assumed to be totally ordered, i.e., $\forall t \in \mathbb{U}_{time} : 0 \leq t \leq \infty$. For convenience, $0$ denotes the start time (or missing timestamp), and $\infty$ denotes the end time of the process. In practice, the timestamps used in this thesis follow the ISO 8601 format: \texttt{YYYY-MM-DD\textit{T}hh:mm:ss\textit{Z}}.
\end{definition}

For later flattening \acp{ocel} we also define the following universes exclusively used for conventional event logs:

\begin{definition}[Event Log Universes]\label{def:xes-universes}
We define the following pairwise disjoint universes:
	\begin{itemize}
		\item $\mathbb{U}_{ev'} \subseteq \mathbb{U}_{ev}$ is a subset of the universe of events.
		\item $\mathbb{U}_{act} \subseteq \mathbb{U}_{etype}$ is a subset of the universe of event types.
		\item $\mathbb{U}_{case} \subseteq \mathbb{U}_{obj}$ is a subset of the universe of objects.
		\item $\mathbb{U}_{time}$ is the universe of timestamps (with $0 \in \mathbb{U}_{time}$ as the smallest and $\infty \in \mathbb{U}_{time}$ as the largest element).
		\item $\mathbb{U}_{res}$ is the universe of resources. %TODO Check if has to be further specified
	\end{itemize}
\end{definition}

\subsection{Object-Centric Event Log}\label{ssec:ocel}

Since this structure is referenced throughout the thesis, we cite here the formal definition of an Object-Centric Event Log (OCEL) as per the OCEL 2.0 specification \autocite{DBLP:journals/corr/abs-2403-01975}.

\begin{definition}[\ac{ocel}]\label{def:ocel}
An \ac{ocel} is a tuple
\[
L = (E, O, EA, OA, \mathit{evtype}, \mathit{time}, \mathit{objtype}, \mathit{eatype}, \mathit{oatype}, \mathit{eaval}, \mathit{oaval}, E2O, O2O)
\]
with the following components:
\begin{itemize}
  \item $E \subseteq \mathbb{U}_{ev}$ is the set of events.
  \item $O \subseteq \mathbb{U}_{obj}$ is the set of objects.
  \item $\mathit{evtype}: E \rightarrow \mathbb{U}_{etype}$ assigns types to events.
  \item $\mathit{time}: E \rightarrow \mathbb{U}_{time}$ assigns timestamps to events.
  \item $EA \subseteq \mathbb{U}_{attr}$ is the set of event attributes.
  \item $\mathit{eatype}: EA \rightarrow \mathbb{U}_{etype}$ assigns event types to event attributes.
  \item $\mathit{eaval}: (E \times EA) \nrightarrow \mathbb{U}_{val}$ assigns values to event attributes (partial function).
  \item $\mathit{objtype}: O \rightarrow \mathbb{U}_{otype}$ assigns types to objects.
  \item $OA \subseteq \mathbb{U}_{attr}$ is the set of object attributes.
  \item $\mathit{oatype}: OA \rightarrow \mathbb{U}_{otype}$ assigns object types to object attributes.
  \item $\mathit{oaval}: (O \times OA \times \mathbb{U}_{time}) \nrightarrow \mathbb{U}_{val}$ assigns values to object attributes (partial function).
  \item $E2O \subseteq E \times \mathbb{U}_{qual} \times O$ is the set of qualified event-to-object relations.
  \item $O2O \subseteq O \times \mathbb{U}_{qual} \times O$ is the set of qualified object-to-object relations.
\end{itemize}

such that
\begin{align*}
  \text{dom}(\mathit{eaval}) &\subseteq \{ (e, ea) \in E \times EA \mid \mathit{evtype}(e) = \mathit{eatype}(ea) \}, \\
  \text{dom}(\mathit{oaval}) &\subseteq \{ (o, oa, t) \in O \times OA \times \mathbb{U}_{time} \mid \mathit{objtype}(o) = \mathit{oatype}(oa) \}.
\end{align*}
\end{definition}

\subsection{Flattened Event Log}
Due to the object-centric nature of \acp{ocel} it might be necessary to flatten an event log i.e., choose a fixed case notion in order to use existing process mining algorithms. This can be formalized as a function mapping the \ac{ocel} and an included object type onto the resulting conventional event log. During this thesis we call the resulting event log a viewpoint on the corresponding \ac{ocel}.

Therefore we first define a conventional event log, that can be expressed e.g., in the \ac{xes} format \autocite{DBLP:journals/cim/AcamporaVSAGV17}.

\begin{definition}[\ac{xes} Event Log]\label{xes-event-log}
An event is denoted by $e' \in \mathbb{U}_{\mathit{ev'}}$. We define the following projection functions:
	\begin{itemize}
  		\item $\pi_{\mathit{case}}(e') \in \mathbb{U}_{\mathit{case}}$ is the \emph{case} of $e$,
  		\item $\pi_{\mathit{act}}(e') \in \mathbb{U}_{\mathit{act}}$ is the \emph{activity} of $e$,
  		\item $\pi_{\mathit{time}}(e') \in \mathbb{U}_{\mathit{time}}$ is the \emph{timestamp} of $e$,
  		\item $\pi_{\mathit{res}}(e') \in \mathbb{U}_{\mathit{res}}$ is the \emph{resource} of $e$.
	\end{itemize}
	An event log $L'$ is defined as a set of events i.e., $L' \subseteq \mathbb{U}_{\mathit{ev'}}$.
\end{definition}

\begin{definition}[Flattening of \ac{ocel}]\label{def:flatten-event-log}
Let the flattening function $\varphi$ of an \ac{ocel} for any \ac{ocel} $L \in \mathbb{L}$ the universe of all \acp{ocel}, any object type $OT \in \mathbb{U}_{otype}$ and $L' \in \mathbb{L'}'$ the universe of all \ac{xes} logs be defined as $\varphi \in \mathbb{L} \times \mathbb{U}_{otype} \rightarrow \mathbb{L'}'$, such that for all events $e' \in L'$:
	\begin{itemize}
		\item $objtype(\pi_{\mathit{case}}(e')) = OT$ 
		\item $\pi_{\mathit{act}}(e') = evtype(e)$
		\item $\pi_{\mathit{time}}(e') = time(e)$
		\item $\pi_{\mathit{res}}(e') = $
		%TODO as {objectID: qualifier} not good enough, could then also be omitted (solve with above TODO)
	\end{itemize}
	%TODO Finalize and double check formality with using e' and e
\end{definition}